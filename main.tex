\documentclass[12pt]{article}
\usepackage{my_preamble}

% Document Starts
\begin{document}

\maketitle 

\begin{abstract}
  We're studying abstract algebra, specifically groups and rings.
\end{abstract}

\tableofcontents

\pagebreak
% <++>

\section{Operations on Sets}
\subsection{K-Ary Operations}

\begin{itemize}
  \item \(\N\quad +,\cdot\)
  \item \(\Z\quad +,\cdot,-\)
  \item \(\Q\quad +,\cdot,-\)
  \item \(\R\quad +,\cdot,-\)
  \item \(\C\quad +,\cdot,-,x\mapsto\overline x,x\mapsto\sqrt x\)
  \item \(\TM{Vectors}\quad +,\TM{scalar mul}\)
  \item \(\TM{Matricies}\quad +,\TM{scalar mul},\TM{matrix mul}\)
  \item \(\TM{polynomials}\quad +,\cdot\)
\end{itemize}

In abstract algebra, we're iinterested in what notions of "numbers" exists.

The different "types" of numbers really are distinguished by the operations on
them. In this class we'll stick with operating on sets.

\bbox
\begin{defn}[Binary Operations]
  A \TB{binary operation} on a set \(X\) is a function \(b:X\times X\to X\).
\end{defn}
\ebox
\bboxnote
\TB{Note:} we often write binary operators inline (like in Haskell).
\ebox

We could use \(+,\cdot,\times,\div,\otimes,\boxtimes,\oplus,\boxplus,\diamond\)
\begin{verbatim}
FIND \gop in the .tex file to change the operator used. 
/\\newcommand{\\gop}
\end{verbatim}


\bbox
\begin{defn}
  a \TB{k-ary operator} on \(X\) is a func \(f:\underset k{\underbrace{
  X\times\cdots\times X}}\to X\).
\end{defn}
\ebox

\bboxnote
\(x\mapsto\frac1x\) on \(\Q\) isn't a unary operation b/c \(\frac10\) isn't
defined.

\(\Q^\times=\{x\in\Q:x\neq 0\}\) does have the reciprocal as a binary operator,
but not minus.
\ebox

\subsection{Associative Operations}
\newcommand{\gop}[0]{\boxtimes}

\bbox
\begin{defn}
  a binary operator \(\gop\) on \(X\) is \TB{associative} if 
  \[x\gop(y\gop z)=(x\gop y)\gop z,\quad\forall x,y,z\in X\]
\end{defn}
\ebox

\bboxnote
\(+,\cdot\T{ on }\N,\Z\) are associative. \(-:\Z\times\Z\to\Z\) isn't
associative. Neither is \(\div:\Q^\times\times\Q^\times\to\Q^\times\). Function
composition is associative.
\ebox

\bbox
\begin{defn}[(Informal) Bracketing]
  Let \(\gop\) be a bin operator on a set \(X\). A \TB{bracketing} of a seq
  \(a_1,\dots,a_n\in X\) is a way of inserting brackets into
\end{defn}
\[a_1\gop\cdots\gop a_n\st\T{the expression can be evaluated}\]
\ebox


\bbox
\begin{defn}[Bracketing]
  A \TB{bracket} of \(a_1,\dots,a_n\) is
  \begin{equation*}
    \begin{split}
      n=1 &:\TM{word}\;a_1\\
      n>1 &:(w_1\gop w_2)\T{ where}\\
          &w_1\is\TM{bracket}\T{ of }a_1,\dots a_k\\
          &w_2\is\TM{bracket}\T{ of }a_{k+1},\dots,a_n
    \end{split}
  \end{equation*}
\end{defn}
\ebox

\begin{tcolorbox}[colback=red!20!white,colframe=red]
  \begin{verbatim}
  data Bracket t = Number t | Branch (Bracket t) (Bracket t) 
  evalBracket :: (t -> t -> t) -> Bracket t -> t
  evalBracket fn aseq =
    case aseq of
      Number x            -> x
      Branch left' right' -> fn (evalBracket fn left')
                                (evalBracket fn right')\end{verbatim}
\end{tcolorbox}


\bbox
\begin{prop}
  a binary operation \(\gop\) on \(X\) is associative \TB{iff} for every
  seq \(a_1,\dots,a_n,\,n\ge 1\), every bracketing of \(a_1,\dots a_n\) 
  evaluates to the same elem of \(X\).
\end{prop}
\ebox

\bboxproof
\begin{proof}
  \((\limplies)\) Take \(n=3\). Then
  \[(a\gop b)\gop c=a\gop(b\gop c),\;\forall a,b,c\in X\]

  \((\implies)\) Proof by induction.

  Base Case: \(n=1\). Every bracketing of a word evaluates to that same word.
  
  Assume proposition is true for \(n<k\), where \(k>1\). Let \(a_1,\cdots,a_k
  \in X\). If \(w\) is a bracketing of \(a_1,\dots,a_k\) then
  \(w=(w_1\gop w_2)\), where
  \(w_1\) is a bracketing of \(a_1,\dots,a_l\) and \(w_2\) is a bracketing of
  \(a_{l+1},\dots,a_k\).
  \[w_1=(\cdots(a_1\gop a_2)\gop\cdots)\gop a_l\]
  \[w_2=(a_{l+1}\gop(\cdots(a_{k-1}\gop a_k)\cdots))\]
  \begin{equation*}
    \begin{split}
      w&\overset{\T{in }X}=w_1\gop w_2\\
       &=(A\gop a_l)\gop w_2\\
       &=A\gop(a_l\gop w_2)\T{ by assoc.}\\
      \cdots&= a_1\gop(\cdots(a_{k-1}\gop a_k)\cdots)
    \end{split}
  \end{equation*}
  Hence any 2 bracketings of \(a_1,\dots,a_k\) evaluate to \(a_1\gop
  (\cdots(a_{k-1}\gop x_k)\cdots)\). By induction, the prop holds.
\end{proof}
\ebox

% lecture 2

\bboxnote
\begin{nota}[Associativity makes Brackets Pointless]\label{nota:bad_bracks}
  Since \(\gop\) is associative, brackets become
  redundant. \(a\gop b\gop c:=
  a\gop(b\gop c)\)
\end{nota}
\ebox


\bbox
\begin{defn}[Commutative] \label{defn:comm}
  \((\gop):X\times X\to X\) is \TB{commutative} (or "abelian") if
  \[a\gop b=b\gop a,\forall a,b\in X\]
\end{defn}
\ebox

\bboxex
\(+,\cdot\T{ on }\R,\C,\Q,\Z\) are commutative. 
\ebox

\bboxex
\(+\T{ on }M_{n\times m}\is\TM{comm}\). \(\TM{matrix\,mul}\T{ on }M_n
\not\is\TM{comm}\).
\ebox

We're much more focused on associative operators as opposed to commutative.

We cover 2 Topics:
\begin{enumerate}
  \item \TB{Group Theory:} a single associative op w/ some additional properties
  \item \TB{Ring Theory:} 2 associative op that behave "like" + \& \(\cdot\).
\end{enumerate}


\bbox
\begin{defn}[Identity]\label{defn:iden}
  An identity for a given bin op \(\gop\) is a element \(e\in X\st e\gop
  x=x\gop e=x,\;\forall x\in X\).
\end{defn}
\ebox

\bboxex
\(0\) is an identity for \(+\) on \(\Z,\R,\Q,\cdots\).
\(1\) is an identity for \(\cdot\) on \(\Q\)
\ebox


\bbox
\begin{lem}[Uniqueness of Identity]\label{lem:uniq_of_iden}
  If \(e\) and \(e'\) are identities for \(\gop\) on \(X\), then \(e=e'\).
\end{lem}
\ebox

\bboxproof
\begin{proof}
    \(e=e\boxtimes e'=e'\)
\end{proof}
\ebox

\bbox
\begin{defn}[Inverse]\label{defn:inv}
  let \(\gop\) be a bin op on \(X\) with iden \(e\). Let \(x\in X\). 
  \(y\in X\) is a
  \begin{enumerate}
    \item \TB{left inverse} for \(X\) if \(y\gop x=e\),
    \item \TB{right inverse} for \(X\) if \(x\gop y=e\),
    \item and an \TB{inverse} for \(X\) if \(x\gop y=e=y\gop x\).
  \end{enumerate}
\end{defn}
\ebox

\bbox
\begin{lem}[Associtivity implies Uniqueness of Inverses]\label{lem:assoc_then_uniq_inv}
  Suppose we have \((\gop)\is\TM{assoc}\). If \(y_L\T{ and }y_R\) are
  left and right inverses of x, then \(y_L=y_R\).
\end{lem}
\ebox

\bboxproof
\begin{proof}
  \begin{equation*}
    \begin{split}
      (y_L\gop x)\gop y_R &=e\gop y_R\\
                          &=y_R\\
      (y_L\gop x)\gop y_R&=y_L\gop(x\gop y_R)\quad\T{(by assoc)}\\
                         &=y_L\gop e\\
                         &=y_L
    \end{split}
  \end{equation*}
\end{proof}
\ebox

\bboxnote
\TB{Consequences:} \(x\) is invertable \TB{iff} it has a left and right
inverse.
\ebox

\bboxnote
\TB{Note:} is is possible to be left invertable but not right invertable, and
vice verse.
\ebox

\bboxex
\(\N=\{1,2,\dots\},+\) has no invertable elements.

\((\Z,+)\) has every element invertable.

\((\Z,\cdot)\) has only \(\{\pm 1\}\) as invertable.

\((\Q,\cdot)\) has \(\Q^\times\) as invertable.
\ebox

\bboxnote
\begin{nota}[Inverse]\label{nota:inv}
  if \(x\) is inivertable, and has a uni inv, then we denote it
  \(x^{-1}\).
\end{nota}
\ebox

\bbox
\begin{lem}[Properties of the Inverse]\label{lem:prop_of_inv}
  Let \((\gop)\is\TM{assoc}\) w/ id \(e\). 
  \begin{enumerate}
    \item \(e\) is invertable, \(\inv e=e\). \bboxproof\begin{proof} \(e\gop e=e\)\end{proof}\ebox
    \item if \(a\) is invertable, then so is \(\inv a\nd \inv{(\inv a)}=a\).
    \item if \(a\nd b\is\TM{invertable}\implies\inv{(a\gop b)}=\inv b\gop\inv a\)
      \bboxproof
      \begin{proof}\(a\gop b\gop\inv b\gop\inv a=a\gop e\gop\inv a=e\) 
        Similiar in reverse.
      \end{proof}
      \ebox
    \item \(a\) is invertable \TB{iff}
      \begin{equation*}
        \begin{split}
          a\gop x&=b\\
          y\gop a&=b
        \end{split}
      \end{equation*} both have uniq sols \(\forall b\in X\).
      \bboxproof
      \begin{proof}
        \((\implies)\) Assume \(a\) is invertable. Then
        \begin{equation*}
          \begin{split}
            x&=\inv a\gop b\\
            y&=b\gop\inv a
          \end{split}
        \end{equation*}

        \((\limplies)\) Assume the system has a uni \(X\nd Y,\;\forall B\in X\).
        Let \(b=e\), where \(e\) is the identity of \((X,\gop)\).
        \begin{equation*}
          \begin{split}
            a\gop x&=e\\
            y\gop a&=e\\
            \implies x&=\inv a_R\\
            \implies y&=\inv a_L\\
            (\gop)\is\TM{assoc}&\implies \inv a_R=\inv a_L=\inv a\\
                               &\implies a\T{ is invertable}
          \end{split}
        \end{equation*}
      \end{proof}
      \ebox
  \end{enumerate}
\end{lem}
\ebox

% lecture 3


\bbox
\begin{lem}[Cancellation Property]\label{lem:cancel_prop}
  Let \(\gop\) be an assoc bin op on \(X\) w/ id \(e\). 

  If it has a left inverse and \(a\gop u=a\gop v\implies u=v\).
  Vice Versa.
\end{lem}
\ebox

\bboxproof
\begin{proof}
  It should be taken as an axiom.
\end{proof}
\ebox

\section{Groups}
\subsection{Definitions}
\bbox
\begin{defn}[Group]\label{defn:group}
  a group is a pair \((G,\gop)\) where \(G\) is a set and \(\gop\)
  is an assoc bin op on \(G\), w/ and id \(e\), s/t every elem of \(G\) is 
  invertable.
\end{defn}
\ebox

\bbox
\begin{nota}[Multiplicative Notation]\label{nota:mul_nota}
  if the op is clear, we'll usually just write \(G\) instead of \((G,\gop)\).

  We often use \((\cdot)\) as the default symbol for the operation on a group,
  or even just writing \(g\cdot h=gh\). The identity can be denoted by
  \(e,e_G,1,1_G\). We use \(\inv a\) for the inverse of \(a\). This is called
  \TB{multiplicative notation}.
\end{nota}
\ebox

\bbox
\begin{defn}[Abelian Group]\label{defn:abel_group}
  A group \((G,\gop)\) is \TB{abelian} if \(\gop\) is abelian (commutative).
\end{defn}
\ebox

For abelian groups, we often use additive notation.

\((+),\mathrm{id}\is 0 \T{ or } 0_G, \mathrm{inv}(a)=a\).

\bboxex
\begin{enumerate}
  \item \(\Z^+\) is an abelian group (under \(+\)).
    \[(+)\is\TM{assoc},\;\mathrm{inv}(a)=-a,\;\mathrm{id}(+)=0,\;+\is\TM{bin\;op}\]
    Note that this is true for \(\Q^+=(\Q,+),\;\R^+\).
  \item \(\Z^\cdot\) isn't a group; every element in \(\Z\) isn't invertable under
    \((\cdot)\)
  \item We know that \(|\Z|=|\Q|\). Let \(\phi\is\TM{bij}:\Z\to\Q\).
    Define an operator on \(\Z\) by \(a\gop b=\inv \phi(\phi(a)+\phi(b))\).
    \((\Z,\gop)\) is an (abelian) group. (\TB{Ex:} \(1\gop 2=8\))
\end{enumerate}
\ebox

\bbox
\begin{lem}
  Let \(\gop\is(\T{assoc},\T{bin op on }M,\mathrm{id}\is e)\),
  \(G:=\{g\in M:g\is\TM{invertable\;wrt\;\boxtimes}\}\). Then
  \(G\) is a group w/ \(g\cdot h:=g\gop h\).
\end{lem}
\ebox

\bboxproof
\begin{proof}
  Hmk.
\end{proof}
\ebox

The smallest possible group is called the trivial group, and it has one
element, \(\{e\},\;ee=e\).

\begin{enumerate}
  \item invertability
  \item identity
  \item closure of \((\gop)\)
  \item assoc of \((\gop)\).
\end{enumerate}


\bboxex
\[\Q^\times=\{a\in\Q:a\neq0\}\]
\[\R^\times=\{a\in\R:a\neq0\}\]
Both are groups under multiplication (bad notation considering \(\R^+\)
is a group but \(\Q^\times\) is a set. I assume \(\Q^\cdot\) is a group, equal
to \((\Q^\times,(\cdot))\)).
\ebox

\bbox
\begin{coro}
  Let \(X\) be a set, and let \(S_X\) be the set of functions
  \(\{f:X\to X:f\is\TM{invertable}\}=\{f\in\mathrm{Fun}(X,X):f\is\TM{inv}\}\).

  Then \(S_X\) is a group under function composition.
\end{coro}
\ebox

\bbox
\begin{defn}[Permutation Group]\label{defn:perm_group}
  \(X:=\{1,\dots,n\}\). \(S_X\) is called the permutation group of rank \(n\),
  and is denoted \(S_n\).
  \bboxnote
  \[\Sigma:=\{\sigma\in\mathrm{Fun}(X,X):\sigma\is\TM{bij}\}\]
  \[S_X:=(\Sigma,\circ)\]
  \ebox
\end{defn}
\ebox

\bbox
\begin{defn}[Order]\label{defn:order}
  The order of a group \(G=(E,\gop)\) is \(|G|=|E|\), where \(E\) is finite. If
  \(E\) is infinite, we'll say \(|G|=+\infty\).
\end{defn}
\ebox


\bboxex
\[|S_n|=|(\Sigma,\circ)|=n!\]
\ebox

% lecture 4

\((M,\gop)\is\TM{monoid}
\implies(M|_{\TM{inv}},\gop)\is\TM{group}\)

\bboxex
\TB{Example of above.} \(M_n\F:=M_{n\times n}(\F)\).

\((M_n\F,\cdot)\is\TM{monoid}\), so
\(\left(M_n\F|_{\TM{inv}},\cdot\right)
\is\TM{group}\)
\ebox


\bboxnote
\begin{nota}[General Linear Group]\label{nota:gen_L_group}
  \(\left(M_n\F|_{\TM{int}},\cdot\right)\)
  is called the \TB{general linear group}
  (over \(\F\)), denoted
  \[\mathrm{GL}_n\F\]
\end{nota}
\ebox

\subsection{Dihedral Groups}
\bbox
\begin{defn}[N-Gon]\label{defn:n_gon}
  Let \(\poly_n:n\ge3\) denote the regular
  n-gon, with verticies
  \[v_k=\left(
      \cos\frac{2\pi k}n,
      \sin\frac{2\pi k}n
  \right):0\le j\le n\;
\T{ noting }v_n=v_0\]
\end{defn}
\ebox

\bbox
\begin{defn}[An N-Gon Symmetry]
  A \TB{symmetry of the n-gon} is an
  elem \(T\in\TR{GL}_2\R\st
  T(\poly_n)=\poly_n\)
\end{defn}
\ebox

\bbox
\begin{defn}[Dihedral Group]\label{defn:dihedral_group}
  The set of symmetries of \(\poly_n\) is called the dihedral group of rank \(n\), denoted \(D_{2n}\).
\end{defn}
\ebox

\bbox
\begin{lem}[Dihedral GROUP]
  \(D_{2n}\) is a group under matrix multiplication.
\end{lem}
\ebox

\bboxproof
\begin{proof}
  Later, in section subgroups.
\end{proof}
\ebox

What are the elems of \(D_{2n}\)?
\(\TR{Id}(D_{2n})=I_2\).

Rotation s by \(\frac{2\pi}n\) radians are elems, and \(s(v_i)=v_{i+1},\;\forall i=0,\dots,n-1\).

Reflection along the \(x\) axis is an elem, so \(r(v_i)=v_{n-i}\).


\bbox
\begin{defn}[Group Power]\label{defn:group_power}
  \(\,\)

  \(g^n:=\underset n{\underbrace{g\cdot g\cdot \cdots \cdot g}},\;n\ge 0\).

  \(g^{-n}:=\underset n{\underbrace{\inv g\cdot \inv g\cdot\cdots\cdot\inv g}}\)

  \(g^0:=e=\TR{Id}(G)\).
  \bboxnote \TB{Note.}
  \(g^{-n}=(\inv g)^n\) and
  \(g^{-n}g^n=e\) Prove the following:
  \[g^ng^m=g^{n+m}\]
  \[(g^n)^m=g^{nm}\]
  \ebox
\end{defn}
\ebox

All this also has additive notation.
\(ng=g+\cdots+g\) and
\((-n)g=\underset n{\underbrace{(-g)+\cdots+(-g)}}\)


\bboxnote
\TB{Note.} 
\[(gh)^n\neq g^nh^n\]
\ebox


\bbox
\begin{defn}[Order of an Element]\label{defn:order_of_and_elem}
  The \TB{order} \(g\in G\) is denoted
  \[|g|:=\TR{min}(\{k\ge 1:gk=e\}\cup\{+\infty\})\]
\end{defn}
\ebox

\bboxex
\TB{Example:} \(|e|=1,\;|g|=1\iff g=e\).
\(\Z^+,\;|1|=+\infty\). 
\((\Z\backslash n\Z,+)\;|[1]|=n\T{ b/c }
n\cdot[1]=0\)
\ebox

\bbox
\begin{lem}[Properties of Order]\label{lem:prop_of_order}
  \begin{enumerate}
    \item \(g^n=e\implies g^{n-1}\cdot g=e\implies g^{n-1}=g^{-1}\)
    \item \(g^n=e\iff(g^n)^{-1}=e\)
  \end{enumerate}
  \(\implies|\inv g|=|g|\)
\end{lem}
\ebox


\bboxex
\TB{Example}
\[-[1]=(n-1)[1]=[n-1]\]
\ebox

Back to dihedral groups.

\(D_{2n},\;|s|=n\equiv s^n(\poly_n)=\poly_n\).
\(|r|=2\equiv r^2(\poly_n)=\poly_n\).

So \(e,s,s^2,\dots,s^{n-1}\in D_{2n}\),
and \(r,sr,s^2r,\cdots,s^{n-1}r\in D_{2n}\).


\bbox
\begin{prop}[Dihedral Group Explicit Classification]\label{prop:dihedral_group_explicit_classification}
  \(D_{2n}=
  \{s^i:0\le i<n\}\cup
  \{s^ir:0\le i<n\}\T{ and }
  |D_{2n}|=2n\)
\end{prop}
\ebox


\bboxproof
\begin{proof}
  \(S,T\in D_{2n}\). So \(S,T\) are linear operations. So if
  \[S(v_0)=T(v_0)\]
  \[S(v_1)=T(v_1)\]
  \[\implies S=T\]
  following if we treat \(v_0,v_1\) as basic vectors.

  \TB{Claim 2:} \(T\in D_{2n}\implies\)
  \[(T(v_0),T(v_1))\in\{(v_i,v_{i+1}):0\le i\le n-1\}
  \cup\{(v_{i+1},v_i):0\le i\le n-1\}=:V\]
  \bboxproof
  \begin{proof}
    \(v_0,v_1\) have to be sent to adj verticies (by picture lol, 
    although we could use a "convexity" arguement).

    $(s^i(v_0),s^i(v_1))$ $=(v_i,v_{i+1})$
    $(r(v_0),r(v_1))$
    $=(v_0,v_{n-1})$

    $(s^ir(v_0),s^ir(v_1))$ $=(s^i(v_0),s^i(v_{n-1})$ $(v_i,v_{i-1})$
  \end{proof}
  \ebox

  \TB{Claim 3:} \(\phi:D_{2n}\to V:\phi(T)=(T(v_0),T(v_1))\) is a 
  bijection. By claim 1, it's a injection, and by the calculation above,
  it's a surjection. Finally, \(2n=|V|\T{ and }D_{2n}=|V|\implies 2n=|D_{2n}|\).

  So \(\{s^i:0\le i\le n-1\}\cup\{s^ir:0\le i\le n-1\}=D_{2n}\).

  Also \(rs(v_0)=r(v_1)=v_{n-1}\), \(rs(v_1)=v_{n-2}\) so
  \(rs=s^{n-1}r=\inv sr\).
\end{proof}
\ebox

\subsection{Subgroups}


\bbox
\begin{defn}[Subgroup]\label{defn:subgroup}
  Let \(G\) be a group. \(H\mathbin{:\subseteq} G\) is a \TB{subgroup} if
  \begin{enumerate}
    \item \(\forall g,h\in H,\ g\cdot h\in H\)
    \item \(g\in H\implies\inv g\in H\)
    \item \(e_G\in H\)
  \end{enumerate}
\end{defn}
\ebox

\bboxnote
\begin{nota}[Subgroup]\label{nota:subgroup}
  \(H\le G\iff H\T{ is a subgroup of }G\).
\end{nota}
\ebox

\bbox
\begin{prop}[A Subgroup is a Group]\label{prop:a_subgroup_is_a_group}
  \(H\le G\implies G\is\TM{group}\), with \(\cdot_H:H\times H\to H\).
\end{prop}
\ebox

\bboxproof
\begin{proof}
  First, \(\cdot_H\) is well defined b/c \(H\) is clsd under \(\cdot_G\).

  Next, \(e_G\) is an identity for \(\cdot_H\).

  \(\cdot_H\) is assoc b/c \(\cdot_G\) is assoc.

  Finally, and elem is inv b/c it has an inverge in H wrt \(\cdot_G\).
\end{proof}
\ebox

\bboxex
\begin{exam}\(\ \)
  \[\Z^+\le\Q^+\le\R^+\le\C^+\]
  \[\N^+\not\le\Z^+\]
\end{exam}
\ebox


\bboxex
\begin{exam}[The Dihedral Group is a Subgroup of the General Linear Group]\label{exam:the_dihedral_group_as_a_subgroup_of_the_general_linear_group}
  \[D_{2n}\le\TR{GL}_2\R\]
\end{exam}
\ebox

\bboxex
\begin{exam}
  \(\Q_{>0}\le\Q^\times\)
  \[\TB{2: }\inn{s}=\{s^i:0\le i<n\}\le D_{2n}\]
\end{exam}
\ebox
\bboxproof
\begin{proof}[Proof of 2]
  \(\ \)
  \begin{enumerate}
    \item \(s^i\cdot s^j=s^{i+j}=s^{an+k}=(s^n)^a\cdot s^k=e^a\cdot s^k=s^k\)
    \item \(\inv g=g^{n-i},\ \inv{(g^0)}=g^0\)
    \item \(e=s^0=e\)
  \end{enumerate}
\end{proof}
\ebox


\bboxex
\begin{exam}
  \(m\Z:=\{mk:k\in\Z\}\le\Z^+\).
  \bboxproof
  \begin{proof}
    \(mj_1+mk_2=m(k_1+k_2)\in m\Z\).
    \(-mk=m(-k)\in m\Z\).
    \(0=m0\in m\Z\).
  \end{proof}
  \ebox

\end{exam}
\ebox

\bboxex
\begin{exam}
  \(G\le G\) and \(\{e_G\}\le G\).
\end{exam}
\ebox

\bbox
\begin{defn}[Proper Subgroup]\label{defn:proper_subgroup}
  \(H\le G\) is \TB{proper} if \(H\neq F\), denoted \(H<G\).
\end{defn}
\ebox
"nontrivial proper subgroup..." \(\{e\}\neq H\neq G\)


\bbox
\begin{prop}
  Let \(H\subseteq G\) be a subset of a group \(G\). Then \(H\le G\) iff
  \begin{enumerate}
    \item \(H\neq\emptyset\) and
    \item \(g,h\in H\implies g\cdot\inv h\in H\).
  \end{enumerate}
\end{prop}
\ebox

\bboxproof
\begin{proof}
  \((\implies)\) clear.

  \((\limplies)\) Suppose \((a)\) and \((b)\) hold. By \((a),\ H\ni g\).
  By \((b)\), \(e=g\cdot\inv g\in H\).
  If \(g,h\in H\), then \(\inv h\in H\), so \(g\cdot\inv{(\inv h)}\in H\),
  but \(g\cdot\inv{(\inv h)}=g\cdot h\in H\).
\end{proof}
\ebox

\bbox
\begin{exam}[Subspaces are Subgroups of the Additive Vector Space Group]
  If \(W\) is a subspace of vector space \(V\), the \(W\le V^+\).
\end{exam}
\ebox

\bboxproof
\begin{proof}[Check]
  \(0\in W\) so nonempty.

  \(v,w\in W\implies v-w\in W\), so W is a subgroup.
\end{proof}
\ebox

\bbox
\begin{prop}
  Suppose \(G\is\TM{group}\) and \(H\subseteq G\) is finite. Then
  \(H\subseteq G\) \TU{iff}
  \begin{enumerate}
    \item \(H\neq\emptyset\)
    \item \(g\cdot h\in H\)
  \end{enumerate}
\end{prop}
\ebox

\bboxproof
\begin{proof}
  \((\implies)\) Clear.

  \((\limplies)\) Assume (1.) and (2.) hold. So \(g\in H\). By
  induction, \(g^n\in H\ \forall n\ge 1\).

  B/c \(H\) is finite, \(g^1,g^2,g^3,\dots\), repeats. So
  \(g^i=g^j\) fore some \(1\le i<j\).

  But now we know \(g^{j-i}=e\in H\) noting that \(j-i\ge 1\).

  We now know \(g^n\in H\) for \(n\ge 0\). Since \(g^{j-i}=e\implies
  g^{j-i-1}\cdot g=e\implies\inv g=g^{j-i-1}\). Since \(j-i-1\ge 0\), 
  \(\inv g\in H\).

  So if \(g,h\in H\), then \(\inv h\in H\), then \(g\inv h\in H\), so \(H\)
  is a subgroup.
\end{proof}
\ebox

\bboxnote
\TB{Aside.} the Set of subgroups of \(G\) form a \TB{lattice}.
\ebox

\bbox
\begin{prop}
  Suppose \(\FF\) is a nonempty set of subgroups of \(G\). Then
  \[K=\bigcap_{H\in\FF}H\]
  is a subgroup.
\end{prop}
\ebox


% <++>
\end{document}


























% scrolloff buffer
